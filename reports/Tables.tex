\documentclass[]{article}
\usepackage{lmodern}
\usepackage{amssymb,amsmath}
\usepackage{ifxetex,ifluatex}
\usepackage{fixltx2e} % provides \textsubscript
\ifnum 0\ifxetex 1\fi\ifluatex 1\fi=0 % if pdftex
  \usepackage[T1]{fontenc}
  \usepackage[utf8]{inputenc}
\else % if luatex or xelatex
  \ifxetex
    \usepackage{mathspec}
  \else
    \usepackage{fontspec}
  \fi
  \defaultfontfeatures{Ligatures=TeX,Scale=MatchLowercase}
\fi
% use upquote if available, for straight quotes in verbatim environments
\IfFileExists{upquote.sty}{\usepackage{upquote}}{}
% use microtype if available
\IfFileExists{microtype.sty}{%
\usepackage{microtype}
\UseMicrotypeSet[protrusion]{basicmath} % disable protrusion for tt fonts
}{}
\usepackage[bottom=1.5cm,top=1.5cm,left=1.5cm,right=1.5cm]{geometry}
\usepackage{hyperref}
\hypersetup{unicode=true,
            pdfborder={0 0 0},
            breaklinks=true}
\urlstyle{same}  % don't use monospace font for urls
\usepackage{longtable,booktabs}
\usepackage{graphicx,grffile}
\makeatletter
\def\maxwidth{\ifdim\Gin@nat@width>\linewidth\linewidth\else\Gin@nat@width\fi}
\def\maxheight{\ifdim\Gin@nat@height>\textheight\textheight\else\Gin@nat@height\fi}
\makeatother
% Scale images if necessary, so that they will not overflow the page
% margins by default, and it is still possible to overwrite the defaults
% using explicit options in \includegraphics[width, height, ...]{}
\setkeys{Gin}{width=\maxwidth,height=\maxheight,keepaspectratio}
\IfFileExists{parskip.sty}{%
\usepackage{parskip}
}{% else
\setlength{\parindent}{0pt}
\setlength{\parskip}{6pt plus 2pt minus 1pt}
}
\setlength{\emergencystretch}{3em}  % prevent overfull lines
\providecommand{\tightlist}{%
  \setlength{\itemsep}{0pt}\setlength{\parskip}{0pt}}
\setcounter{secnumdepth}{0}
% Redefines (sub)paragraphs to behave more like sections
\ifx\paragraph\undefined\else
\let\oldparagraph\paragraph
\renewcommand{\paragraph}[1]{\oldparagraph{#1}\mbox{}}
\fi
\ifx\subparagraph\undefined\else
\let\oldsubparagraph\subparagraph
\renewcommand{\subparagraph}[1]{\oldsubparagraph{#1}\mbox{}}
\fi

%%% Use protect on footnotes to avoid problems with footnotes in titles
\let\rmarkdownfootnote\footnote%
\def\footnote{\protect\rmarkdownfootnote}

%%% Change title format to be more compact
\usepackage{titling}

% Create subtitle command for use in maketitle
\newcommand{\subtitle}[1]{
  \posttitle{
    \begin{center}\large#1\end{center}
    }
}

\setlength{\droptitle}{-2em}
  \title{}
  \pretitle{\vspace{\droptitle}}
  \posttitle{}
  \author{}
  \preauthor{}\postauthor{}
  \date{}
  \predate{}\postdate{}

\fontsize{8}{20}
\usepackage{lscape}
\pagenumbering{gobble}

\begin{document}

\subsection{Tables}\label{tables}

\begin{table}[ht]
\centering
\caption{Terms used throughout this paper and to classify components of lifespan validated in age determination studies. An example of each term is given for New Zealand porbeagle sharks (Francis \emph{et al.} 2007).\newline} 
\begin{tabular}{p{4cm}p{9cm}p{2.5cm}}
  \toprule
Term & Description & NZ Porbeagle \\ 
  \midrule
Maximum validated age & The oldest individual for which growth zones have been validated. Or, where age underestimation is reported, the age to which growth zone counts are seemingly valid & 20 years \\ 
  Apparent longevity & The oldest individual based on unvalidated growth zone counts & 38 years \\ 
  True longevity & The minimum longevity of the species where age has been shown to be underestimated, and where it exceeds maximum apparent age & 65 years \\ 
  Validated ages & The ages over which growth zones have been confirmed to be a reliable indicator of age & 0 - 20 years \\ 
  Uncertain ages & Any ages for which growth zones have yet to be validated. Also, where age underestimation occurs, the ages between the maximum validated age and the apparent longevity, which are effectively uncertain unless age underestimation is corrected for & 20 - 38 years \\ 
  Underestimated ages & The difference between true longevity and apparent longevity & 38 - 65 years \\ 
   \bottomrule
\end{tabular}
\end{table}

\newpage

\begin{landscape}
\begin{table}[ht]
\centering
\caption{Evidence for age underestimation in bomb carbon dating and chemical marking age validation studies of sharks and rays. $n$ is sample size, $A_{Max}$ is longevity, $\Delta_{Mean}$ and $\Delta_{Max}$ are the mean and maximum differences between true and apparent age in individuals where age underestimation was detected. Regions: AUS/NZ - Australia and New Zealand; NEA - northeast Atlantic; NEP - northeast Pacific; NWA - northwest Atlantic; SA - South Africa.\newline} 
\begin{tabular}{lllp{1.2cm}p{1.2cm}p{0.9cm}p{0.9cm}p{0.9cm}lp{8.5cm}}
  \toprule
Species & Region & $\textit{n}$ & Validated ages (yrs) & Apparent $A_{Max}$ (yrs) & True $A_{Max}$ (yrs) & $\Delta_{Mean}$ (yrs) & $\Delta_{Max}$ (yrs) & Evidence & Rationale \\ 
  \midrule
\multicolumn{10}{l}{Method: Bomb carbon dating}\\
\textit{Alopias vulpinus} \textsuperscript{16} & NWA &   3 & 0-14 &  20 &  38 &  10 &  18 & Likely & Phase-shifted $^{14}$C signature \\ 
  \textit{Carcharadon carcharias} \textsuperscript{9} & NWA &   8 & 0-44 &  52 &  73 &  14 &  21 & Likely & Phase-shifted $^{14}$C signature \\ 
  \textit{Carcharadon carcharias} \textsuperscript{1} & NEP &   8 & 0-7 &  18 &  37 &  12 &  20 & Likely & Phase-shifted $^{14}$C signature. Re-analysis of Kerr \textit{et al.} (2006) \\ 
  \textit{Carcharhinus obscurus} \textsuperscript{15} & NWA &   8 & 0-11 &  23 &  42 &  17 &  19 & Likely & Phase-shifted $^{14}$C signature \\ 
  \textit{Carcharhinus plumbeus} \textsuperscript{2} & NWA &   5 & 0-10 &  27 &  33 &   8 &  11 & Likely & Phase-shifted $^{14}$C signature \\ 
  \textit{Carcharias taurus} \textsuperscript{17} & NWA &   8 & 0-12 &  22 &  34 &  12 &  12 & Likely & Phase-shifted $^{14}$C signature \\ 
  \textit{Carcharias taurus} \textsuperscript{17} & SA &   2 & 0-14 &  23 &  40 &  19 &  20 & Likely & Phase-shifted $^{14}$C signature \\ 
  \textit{Galeocerdo cuvier} \textsuperscript{12} & NWA &   4 & 0-20 &  22 &  &  &  & Possible & Phase-shift noted in one specimen, attributed to ontogenetic diet and depth changes \\ 
  \textit{Galeorhinus galeus} \textsuperscript{11} & AUS/NZ &   9 & 0-14 &  20 &  42 &   9 &  18 & Likely & Phase-shifted $^{14}$C signature \\ 
  \textit{Isurus oxyrinchus} \textsuperscript{3} & NWA &   8 & 0-31 &  31 &  &  &  & Possible & Phase-shift noted in one specimen, attributed to methodological error \\ 
  \textit{Lamna nasus} \textsuperscript{7} & AUS/NZ &  11 & 0-20 &  38 &  65 &  22 &  34 & Likely & Phase-shifted $^{14}$C signature \\ 
  \textit{Leucoraja ocellata} \textsuperscript{13} & NWA &  13 & 0-23 &  23 &  28 &   5 &   5 & Likely & Phase-shifted $^{14}$C signature \\ 
  \textit{Squalus suckleyii} \textsuperscript{4} & NEP &  13 & 0-52 &  80 &  &  &  & Possible & Phase-shift noted in at least one specimen, attributed to methodological error \\ 
   \midrule
\multicolumn{10}{l}{Method: Chemical marking}\\
\textit{Carcharhinus melanopterus} \textsuperscript{5} & AUS/NZ &  11 & 0-8 &  15 &  &   3 &   3 & Likely & Recapture did not form expected number of growth zones \\ 
  \textit{Carcharhinus sorrah} \textsuperscript{10} & AUS/NZ &   8 &  &  13 &  &   2 &   2 & Likely & Recapture did not form expected number of growth zones. Calcein did not mark pregnant females. \\ 
  \textit{Carcharhinus tilstoni} \textsuperscript{10} & AUS/NZ &   1 &  &  15 &  &   3 &   3 & Likely & Recapture did not form expected number of growth zones \\ 
  \textit{Carcharhinus tilstoni} \textsuperscript{6} & AUS/NZ &  10 & 0-3 &  12 &  18 &  &  & Possible & Long-term recapture suggested greater longevity than growth zone counts \\ 
  \textit{Galeorhinus galeus} \textsuperscript{19} & AUS/NZ &  18 & 0-11 &  20 &  42 &  &  & Likely & Frequency of growth zones in sharks >1400mm (mean age ~11) was significantly < 1 \\ 
  \textit{Neotrygon kuhlii} \textsuperscript{18} & AUS/NZ &   3 & 1-5 &  13 &  &   2 &   2 & Likely & Recapture did not form expected number of growth zones \\ 
  \textit{Raja erinacea} \textsuperscript{14} & NWA &  13 & 5-11 &  11 &  &  &  & Possible & Annual growth zones may cease when females reproductively active \\ 
  \textit{Sphyrna tiburo} \textsuperscript{8} & NWA &  24 & 0-10.5 &  18 &  &   2 &   2 & Likely & Recapture did not form expected number of growth zones. Long-term recapture suggested greater longevity than growth zones counts. \\ 
   \bottomrule
\multicolumn{10}{l}{\rule{0em}{2.5ex}\textsuperscript{1}{Andrews and Kerr (2015)}; \textsuperscript{2}{Andrews \textit{et al.} (2011)}; \textsuperscript{3}{Ardizzone \textit{et al.} (2006)}; \textsuperscript{4}{Campana \textit{et al.} (2006)}; \textsuperscript{5}{Chin \textit{et al.} (2013)}; \textsuperscript{6}{Davenport and Stevens (1988)}}\\

\multicolumn{10}{l}{\textsuperscript{7}{Francis \textit{et al.} (2007)}; \textsuperscript{8}{Frazier \textit{et al.} (2014)}; \textsuperscript{9}{Hamaday \textit{et al.} (2014)}; \textsuperscript{10}{Harry \textit{et al.} (2013)}; \textsuperscript{10}{Harry \textit{et al.} (2013)}; \textsuperscript{11}{Kalish and Johnston (2001)}}\\

\multicolumn{10}{l}{\textsuperscript{12}{Kneebone \textit{et al.} (2008)}; \textsuperscript{13}{McPhie and Campana (2009)}; \textsuperscript{14}{Natanson (1993)}; \textsuperscript{15}{Natanson \textit{et al.} (2014)}; \textsuperscript{16}{Natanson \textit{et al.} (2016)}; \textsuperscript{17}{Passerotti \textit{et al.} (2014)}}\\

\multicolumn{10}{l}{\textsuperscript{17}{Passerotti \textit{et al.} (2014)}; \textsuperscript{18}{Pierce and Bennett (2009)}; \textsuperscript{19}{Walker \textit{et al.} (2001)}}\\
\end{tabular}
\end{table}
\end{landscape}

\newpage

\begin{table}[ht]
\centering
\caption{Best fit parameters ($\beta_1$ and $\beta_2$), standard errors (S.E.), and negative log likelihood, ($LL$) for logistic regression models of incidence of age underestimation as function of relative length and age.\newline} 
\begin{tabular}{rrrrrr}
  \toprule
 & $\beta_1$ & S.E. & $\beta_2$ & S.E. & $LL$ \\ 
  \midrule
Length & -36.07 & 13.06 & 41.02 & 14.52 & -8.90 \\ 
  Age & -6.14 & 1.80 & 15.06 & 4.56 & -10.87 \\ 
   \bottomrule
\end{tabular}
\end{table}


\end{document}
