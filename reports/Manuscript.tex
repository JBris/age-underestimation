\documentclass[]{article}
\usepackage{lmodern}
\usepackage{setspace}
\setstretch{2}
\usepackage{amssymb,amsmath}
\usepackage{ifxetex,ifluatex}
\usepackage{fixltx2e} % provides \textsubscript
\ifnum 0\ifxetex 1\fi\ifluatex 1\fi=0 % if pdftex
  \usepackage[T1]{fontenc}
  \usepackage[utf8]{inputenc}
\else % if luatex or xelatex
  \ifxetex
    \usepackage{mathspec}
  \else
    \usepackage{fontspec}
  \fi
  \defaultfontfeatures{Ligatures=TeX,Scale=MatchLowercase}
\fi
% use upquote if available, for straight quotes in verbatim environments
\IfFileExists{upquote.sty}{\usepackage{upquote}}{}
% use microtype if available
\IfFileExists{microtype.sty}{%
\usepackage{microtype}
\UseMicrotypeSet[protrusion]{basicmath} % disable protrusion for tt fonts
}{}
\usepackage[margin=1in]{geometry}
\usepackage{hyperref}
\hypersetup{unicode=true,
            pdftitle={Evidence for systemic age underestimation in shark and ray ageing studies},
            pdfborder={0 0 0},
            breaklinks=true}
\urlstyle{same}  % don't use monospace font for urls
\usepackage{graphicx,grffile}
\makeatletter
\def\maxwidth{\ifdim\Gin@nat@width>\linewidth\linewidth\else\Gin@nat@width\fi}
\def\maxheight{\ifdim\Gin@nat@height>\textheight\textheight\else\Gin@nat@height\fi}
\makeatother
% Scale images if necessary, so that they will not overflow the page
% margins by default, and it is still possible to overwrite the defaults
% using explicit options in \includegraphics[width, height, ...]{}
\setkeys{Gin}{width=\maxwidth,height=\maxheight,keepaspectratio}
\IfFileExists{parskip.sty}{%
\usepackage{parskip}
}{% else
\setlength{\parindent}{0pt}
\setlength{\parskip}{6pt plus 2pt minus 1pt}
}
\setlength{\emergencystretch}{3em}  % prevent overfull lines
\providecommand{\tightlist}{%
  \setlength{\itemsep}{0pt}\setlength{\parskip}{0pt}}
\setcounter{secnumdepth}{0}
% Redefines (sub)paragraphs to behave more like sections
\ifx\paragraph\undefined\else
\let\oldparagraph\paragraph
\renewcommand{\paragraph}[1]{\oldparagraph{#1}\mbox{}}
\fi
\ifx\subparagraph\undefined\else
\let\oldsubparagraph\subparagraph
\renewcommand{\subparagraph}[1]{\oldsubparagraph{#1}\mbox{}}
\fi

%%% Use protect on footnotes to avoid problems with footnotes in titles
\let\rmarkdownfootnote\footnote%
\def\footnote{\protect\rmarkdownfootnote}

%%% Change title format to be more compact
\usepackage{titling}

% Create subtitle command for use in maketitle
\newcommand{\subtitle}[1]{
  \posttitle{
    \begin{center}\large#1\end{center}
    }
}

\setlength{\droptitle}{-2em}
  \title{Evidence for systemic age underestimation in shark and ray ageing
studies}
  \pretitle{\vspace{\droptitle}\centering\huge}
  \posttitle{\par}
\subtitle{Systemic age underestimation of cartilaginous fishes\\
Effects of systemic age underestimation in cartilagnious fishes}
  \author{Alastair Varley Harry\\
Centre for Sustainable Tropical Fisheries and Aquaculture \& College of
Science and Engineering, James Cook University, Townsville, QLD, 4811,
Australia\\
\href{mailto:alastair.harry@gmail.com}{\nolinkurl{alastair.harry@gmail.com}}\\
\emph{Running title: Age underestimation in sharks and rays}}
  \preauthor{\centering\large\emph}
  \postauthor{\par}
  \date{}
  \predate{}\postdate{}

\fontsize{12}{20}
\usepackage{lineno}
\linenumbers
\linenumberfont
\renewcommand\linenumberfont{\normalfont\small\sffamily}
\pagenumbering{gobble}

\begin{document}
\maketitle

\newpage

\subsubsection{Abstract}\label{abstract}

Numerous studies have now demonstrated that the most common method of
ageing sharks and rays, counting growth zones on calcified structures,
can underestimate true age. I reviewed bomb carbon dating (n=15) and
fluorochrome chemical marking (n=44) age validation studies to
investigate the frequency and magnitude of this phenomenon. Age was
likely to have been underestimated in nine of 29 genera and 30\% of the
53 populations studied, including 50\% of those validated using bomb
carbon dating. Length and age were strongly significant predictors of
occurrence, with age typically underestimated in larger and older
individuals. These characteristics suggest age underestimation is likely
a systemic issue associated with the current methods and structures used
for ageing. Where detected using bomb carbon dating, growth zones were
reliable up to 88\% of asymptotic length (\(L_\infty\)) and 41\% of
maximum age (\(A_{Max}\)). The maximum magnitude of age underestimation,
\(\Delta_{Max}\), ranged from five to 34 years, averaging 18 years
across species. Current perceptions of shark and ray life histories are
informed to a large extent by growth studies that assume calcified
ageing structures are valid throughout life. The widespread age
underestimation documented here shows this assumption is frequently
violated, with potentially important consequences for conservation and
management. In addition to leading to an underestimation of longevity,
the apparent loss of population age structure associated with it may
unexpectedly bias growth and mortality parameters. Awareness of these
biases is essential given shark and ray population assessments often
rely exclusively on life history parameters derived from ageing studies.

Keywords: age validation; bomb carbon dating; chondrichthyes;
fluorochrome marking; life history; longevity

\newpage

\tableofcontents

\newpage

\subsection{Introduction}\label{introduction}

The credibility of fish age estimates derived from counts of concentric
growth zones on calcified structures relies on validating the temporal
periodicity of formation over the lifespan of the species (Beamish and
Mcfarlane 1983; Campana 2001). This basic tenet underlies a substantial
amount of modern fisheries science including any stock assessment that
relies on ages derived from these methods, ranging from simple
equilibrium methods such as catch curves or per-recruit analyses to
integrated population dynamics models (Maunder and Punt 2013). Age is
also used in many qualitative risk-based methods (Patrick \emph{et al.}
2010; Hobday \emph{et al.} 2011) and measures of intrinsic vulnerability
(Dulvy \emph{et al.} 2004; Cortés 2016), ultimately informing management
strategies for sustainable exploitation and conservation.

In chondrichthyan fishes (hereafter ``sharks and rays''), where ageing
methodology developed at a much slower pace than teleosts, the first
concerted efforts to determine age (Prince and Pulos 1983; Cailliet
\emph{et al.} 1986) coincided with a renewed focus on age validation
across the wider discipline (Beamish and Mcfarlane 1983). The many
reviews on shark and ray ageing have repeatedly emphasized the
importance of validation to practitioners (Cailliet \emph{et al.} 1986,
2006; Cailliet 1990, 2015; Cailliet and Goldman 2004; Goldman \emph{et
al.} 2012). Accordingly, validation has become an integral and expected
part of most studies, particularly when using the most common method of
counting growth zones on thin sections of vertebrae or spines.

Within the past two decades global ageing of shark and ray populations
has increased rapidly, in parallel with attempts by governments to
implement policies of fisheries management and biodiversity conservation
to meet international commitments (Fischer \emph{et al.} 2012).
Operationalising these policies invariably requires the assessment of
fishing impacts on populations. This has major implications for shark
and ray species, of which an estimated one quarter are threatened by
overfishing (Dulvy \emph{et al.} 2014). In practice, it has meant more
regular stock assessments on target species, and risk assessments on
non-target components of the catch including bycatch and threatened
species. With greater numbers and diversity of sharks and rays being
aged to support these assessments, corresponding studies have
successfully validated part or all of the lifespan in a growing number
of species (Cailliet 2015).

Yet the increasing attempts at validating growth zones have not always
yielded successful or expected results. One issue that has become
increasingly apparent is that of age underestimation, a phenomenon that
appears to result from growth zones ceasing to form or becoming
unresolvable or unreliable beyond a certain size or age. The first
detailed account of age underestimation was in New Zealand porbeagle
sharks (\emph{Lamna nasus}, Lamnidae) (Francis \emph{et al.} 2007).
Using bomb radiocarbon dating, levels of \(\Delta^{14}\textup{C}\) radio
isotopes were found to be `phase-shifted' relative to reference
chronologies. This implied that some sharks were more than twice as old
as originally thought and that growth zones were only reliable in
individuals up to 20 years old. Similar phase-shifts, which have also
been termed `missing time', have now been reported in several other
species studied using bomb carbon dating, including for populations
where well-established and even previously validated ageing protocols
existed (Andrews \emph{et al.} 2011; Passerotti \emph{et al.} 2014).

Age underestimation is not only limited to bomb carbon dating and direct
evidence has also been found using other age validation techniques such
as chemical marking. This includes cases where recaptured individuals
failed to deposit expected annual growth increments after several years
at liberty (Chin \emph{et al.} 2013; Harry \emph{et al.} 2013). Further
circumstantial evidence that growth zone counts underestimate true
longevity is apparent from some long-term tagging studies, most
strikingly in the school shark (\emph{Galeorhinus galeus}, Triakidae)
where tag recaptures of up to 42 years at liberty are more than double
the highest age based on growth zone counts (Moulton \emph{et al.} 1992;
Francis \emph{et al.} 2007).

The full extent of age underestimation in shark and ray ageing studies
is unknown, and its implications have not been investigated. McPhie and
Campana (2009) suggest that improper age assignment due to age
underestimation in skates could have potentially severe management
implications. Chin \emph{et al.} (2013) state that the underestimation
of longevity would impact demographic analyses. Noting the increasing
prevalence in long-lived species, Natanson \emph{et al.} (2014)
questioned the value of vertebral ageing. Despite these concerns, the
\emph{a priori} assumption that annual growth zones form throughout life
persists (Cailliet and Goldman 2004; Francis \emph{et al.} 2007;
Passerotti \emph{et al.} 2014). In light of the mounting evidence for
age underestimation there is a need to re-examine this assumption.

To investigate this issue I reviewed two common approaches for age
validation; chemical marking and bomb carbon dating. Fluorescent dyes
such as oxytetracycline (OTC) and calcein have successfully been used to
mark the calcified structures of tagged and recaptured or captive sharks
and rays since the 1970s (Holden and Vince 1973). The technique involves
injecting a fluorescent marker into the muscle or body cavity of the
fish that is then rapidly incorporated into calcified structures
providing a permanent record of when the individual was caught (Izzo
\emph{et al.} 2007). The use of bomb carbon dating is more recent, but
has been applied to sharks and rays since the 2000s (Campana \emph{et
al.} 2002). This method involves comparing \(\Delta^{14}\textup{C}\)
radio isotope levels in growth zones of unknown age with known-age
reference chronologies (Kalish 1993). The rapid increase in
\(\Delta^{14}\textup{C}\) in marine dissolved inorganic carbon that
occurred globally as a result of nuclear testing during the late 1950s
is preserved as a distinct signature in calcified tissues of organisms
alive during this period, allowing age estimates to be corroborated.
These methods were chosen as they are generally considered to be the
most robust methods for validating periodicity of growth zone deposition
in sharks and rays (Goldman \emph{et al.} 2012) and marine organisms
more generally (Campana 2001). While the two techniques differ
considerably in their applications and are subject to a range of
assumptions (Campana 1999), they are nonetheless the most likely to
detect age underestimation.

In reviewing evidence for age underestimation I (1) quantify its
frequency of detection, (2) quantify its magnitude and the lengths and
ages at which it manifests, and (3) assess how pervasive it is likely to
be in shark and ray ageing studies. Using the New Zealand porbeagle
shark as an example, I formulate hypotheses about the potential impacts
and management implications. I conclude by discussing options for
addressing the issue and the challenges of shark and ray age validation
more generally.

\subsection{Materials and methods}\label{materials-and-methods}

To determine the frequency of age underestimation, I conducted a
comprehensive review of all bomb carbon dating and chemical marking
studies. I primarily restricted my search to peer-reviewed primary
literature, although some influential technical reports or books were
also included. Prior to 2006 a series of literature reviews on age
determination in sharks and rays provided an exhaustive list of
validation studies (Cailliet \emph{et al.} 1986, 2006; Cailliet 1990;
Cailliet and Goldman 2004). Studies from 2006 onward were obtained
through a search of online databases. Long-term studies that resulted in
multiple publications were grouped e.g.~Smith (1984), Kusher \emph{et
al.} (1992), and Smith \emph{et al.} (2003).

Evidence of age underestimation was clear and definitive in some cases,
but subjective in others. Based on the strength of evidence, age
underestimation was classified as either `likely' or `possible'. The
latter category included cases where evidence was ambiguous,
circumstantial in nature, or was retrospectively invoked as an
explanation. For example, interpretation of \(\Delta^{14}\textup{C}\)
signatures used for bomb carbon dating can be confounded by multiple
factors. As diet is the source of carbon in shark and ray vertebrae,
\(\Delta^{14}\textup{C}\) signatures are expected to be phase-shifted
relative to reference chronologies because they feed upon prey with a
\(^{14}\textup{C}\) content different to the surrounding water (Campana
\emph{et al.} 2002). Depth-related attenuation, interspecific
variability, regional differences, ontogeny, and incorrect
interpretation of growth zones may further affect
\(\Delta^{14}\textup{C}\) signatures (Campana 1999; Goldman \emph{et
al.} 2012; Andrews and Kerr 2015), making it necessary to formulate
hypotheses about the potential cause of any observed phase-shift.

Where age validation studies were accompanied by a full age and growth
study, I also attempted to quantify the magnitude of age underestimation
at the population level. Information was collected on the range of ages
validated, the apparent longevity of the species (based on the highest
growth zone counts), and the true longevity of the species (based on age
validation). These data were used to classify the lifespan into three
components; validated ages, uncertain ages, and underestimated ages
(Table 1). The apparent proportion of the lifespan that growth
increments were reliable for was calculated based on these
classifications, assuming that the periodicity of growth zone deposition
was the same for all ages (unless otherwise specified). Where only
lengths of validated individuals were provided, the inverse of the
corresponding growth function was used to determine the approximate
range of ages validated.

Most bomb carbon dating studies reported the details of all individuals
studied. This allowed the lengths and ages in which age underestimation
manifested to be investigated. In the subset of populations where it was
detected, data on the length, sex, apparent age, and true age (Table 1)
were collated for each individual studied. The mean and maximum
difference between true and apparent age, \(\Delta_{Mean}\) and
\(\Delta_{Max}\) respectively, were determined for each population. Age
was divided against the oldest individual reported (\(A_{Max}\)), and
length divided against asymptotic length (\(L_\infty\)) or, if
unavailable, the largest individual reported, to provide two
dimensionless indices of relative age and length. These indices were
used to statistically investigate the incidence of age underestimation
as a function of length and age. Incidence of age underestimation was
modeled separately as a function of relative length and age using
Generalized Linear Models of the form \[Y_i\sim Bernoulli(p_i)\]
\[p_i=logit^{-1}\left (\beta_1+\beta_2\cdot  x_i  \right )\] where the
probability, \(Y_i\), that the age of sample \emph{i} was
underestimated, was approximated by a Bernoulli random variable, \(x_i\)
is relative length or relative age, \(\beta_1\) and \(\beta_2\) are
parameters estimated using maximum likelihood, and \(logit^{-1}(X)\) is
the logistic function \(exp(X)/(exp(X)+1)\). Data were insufficient to
include other explanatory variables (e.g.~species, sex) in the model
structure. All analyses and graphics were undertaken using R (R Core
Team 2016).

\subsection{Results and Discussion}\label{results-and-discussion}

Between 1973 and 2016 there were 58 unique studies including 44 that
used chemical marking and 15 that used bomb carbon dating (a single
study by Andrews \emph{et al.} (2011) used both methods). In total 69
populations of chondrichthyans were studied of which 16, all chemically
marked, were excluded from further analysis because they either failed
to validate any regular temporal periodicity in growth zones, were
purely methodological, or did not provide any details of the lengths or
ages validated. The final data analysed included 20 populations
validated using bomb carbon dating and 33 using chemical marking (see
Supplementary Material for complete reference list and data).

The majority (91\%) of studies were conducted on sharks including 25
species from 17 genera and four orders, with ground sharks
(Carcharhiniformes) and mackerel sharks (Lamniformes) studied most
frequently. Studies on rays and skates consisted of five species from
five genera and two orders. No studies were done on chimaeras. Vertebrae
were the primary ageing structure validated, making up 89\% of studies,
although a small number of studies on dogfish (Squaliformes) and horn
sharks (Heterodontiformes) used spines. Most studies (85\%) counted
growth zones in thin-sections of vertebrae or spines, with the remainder
using whole or half structures for ageing.

\subsubsection{Frequency of detection}\label{frequency-of-detection}

Across both validation methods there were 21 populations identified
where age underestimation was likely or possible (Table 2). Age
underestimation was detected most frequently with bomb carbon dating
where it was inferred from phase-shifted \(\Delta^{14}\textup{C}\)
signatures (Fig. 1). The first phase-shift attributed to age
underestimation was documented in school sharks by Kalish and Johnston
(2001), however the phenomenon did not gain widespread exposure until
its discovery in New Zealand porbeagle sharks (Francis \emph{et al.}
2007). To date, phase-shifts in 10 populations have been directly
attributed to age underestimation. Documented cases involved eight
species from seven genera (Table 2). In general, the large temporal
shift in \(\Delta^{14}\textup{C}\) signature (\(\geq\) five years) led
authors to conclude that age must have been underestimated, as
phase-shifts of this magnitude are inconsistent with other possible
explanations (Goldman \emph{et al.} 2012).

There were also several populations where unexplained, shorter,
phase-shifts were observed. These were distinct from phase-shifts
attributable to the metabolically-derived carbon present in ageing
structures, which generally results in radiocarbon signatures lagging
reference chronologies by a few years (Campana \emph{et al.} 2002).
Ontogenetic dietary and depth changes were invoked to explain the
\(\Delta^{14}\textup{C}\) values in a northwest Atlantic tiger shark
(\emph{Galeocerdo cuvier}, Carcharhinidae) that were phase-shifted
relative to other individuals studied (Kneebone \emph{et al.} 2008). The
authors noted parallels with a study on northeast Pacific white sharks
(\emph{Carcharadon carcharias}, Lamnidae) (Kerr \emph{et al.} 2006),
however the phase-shift in that study was later attributed to age
underestimation (Andrews and Kerr 2015). Methodological error was
thought to explain phase-shifts in some samples of spiny dogfish
(\emph{Squalus} spp., Squalidae) (Campana \emph{et al.} 2006) and
shortfin mako (\emph{Isurus oxyrinchus}, Lamnidae) (Ardizzone \emph{et
al.} 2006) --- the latter species is also further complicated by its
potentially variable growth zone deposition (Natanson \emph{et al.}
2006; Kinney \emph{et al.} 2016). Although not considered in these
cases, underestimation of age by a few years could conceivably be a
causal factor. Excluding these studies where evidence was ambiguous, age
underestimation still likely occurred in a minimum of 50\% of bomb
carbon dating studies (Table 2).

Fewer chemically marked populations exhibited signs of age
underestimation; it occurred in eight populations, including six
considered to be likely and two possible (Fig. 2, Table 2). Likely cases
involved six species from four genera. They included several instances
where chemical marks were visible on the distal margin of vertebrae of
individual specimens following recapture after several years, indicating
that no growth zones had been formed subsequently (Pierce and Bennett
2009; Chin \emph{et al.} 2013; Harry \emph{et al.} 2013; Frazier
\emph{et al.} 2014). Walker \emph{et al.} (2001) also provided
population-level evidence of age underestimation in school sharks, a
finding consistent with the bomb carbon dating by Kalish and Johnston
(2001). Using analysis of variance, the mean number of annual growth
zones was found to be significantly less than one in length classes
\textgreater{}1000 mm, with sharks \textgreater{} 1400 mm producing, on
average, one growth zone every four years.

Age underestimation was possible in a further two chemically marked
populations. Natanson (1993) hypothesized that growth zone deposition
may cease in reproductively active females based on a captive study of
little skates (\emph{Raja erinacea}, Rajidae). Davenport and Stevens
(1988) aged Australian blacktip sharks (\emph{Carcharhinus tilstoni},
Carcharhinidae) to 12 years old, using chemical marking and marginal
increment techniques for validation. A subsequent recapture after 18
years (estimated to have been at least two when tagged) suggests
substantially greater longevity (Harry \emph{et al.} 2013). Similar
long-term recaptures of school shark and bonnethead sharks
(\emph{Sphyrna tiburo}, Sphyrnidae) have also been reported (Francis
\emph{et al.} 2007; Frazier \emph{et al.} 2014). The overall level of
detection in chemical validation studies was lower than that of bomb
carbon dating. Nonetheless it was likely to have occurred in at least
18\% of populations.

The limitations on both methods for detecting age underestimation, and
also the difficulty of comparing detection levels between them, are
highlighted in Figures 1 and 2. Bomb carbon dating was typically done of
a small number of the oldest and largest individuals sampled and the
method validates the entire lifespan of those individuals (Fig. 1). As
age underestimation is generally assumed to occur in older and larger
individuals, it follows that this method is more likely to detect it.
Except in a few cases, chemical marking typically only validated a small
subset of the total lifespan of the population and did not include older
individuals, which are presumably both less abundant and less amendable
to tag and recapture due to their larger size (Fig. 2). As such, age
underestimation may not be expected to be detected in as many chemical
marking studies.

At least two other factors may confound or limit the ability to detect
age underestimation using these methods. The failure of fluorochrome
dyes to actually mark the vertebrae of some individuals was noted in
numerous chemical marking studies. For example, McFarlane and Beamish
(1987) reported that OTC marks failed to form in 34\% of Pacific spiny
dogfish (\emph{Squalus suckleyi}, Squalidae) spines, and that marking
success rates varied between seasons. Walker \emph{et al.} (2001)
attributed this type of failure to either a lack of vertebral
mineralization due to low somatic growth rates or methodological error
such as a failure to properly inject chemicals into the body cavity of
the fish. Since age underestimation is also typically attributed to the
slowing or cessation of growth, these outcomes may be confounded. If
slow growth does indeed limit or stop the incorporation of chemical
markers in calcified structures it implies that they would no longer be
useful as a validation tool. The ability to detect a small
underestimation of age using bomb carbon dating is also unclear since it
would presumably be difficult to distinguish from phase-shifts related
to the metabolically-derived carbon present in shark and ray ageing
structures.

\subsubsection{Magnitude and
manifestation}\label{magnitude-and-manifestation}

The mean magnitude of age underestimation, \(\Delta_{Mean}\), detected
using bomb carbon dating ranged from five years in thorny skates
(\emph{Leucoraja ocellata}, Rajidae) (McPhie and Campana 2009), to 22
years in New Zealand porbeagle sharks (Francis \emph{et al.} 2007),
averaging 13 years across all studies (Table 2). The greatest individual
disparity between true and apparent age, \(\Delta_{Max}\), was also in a
New Zealand porbeagle shark that was underaged by 34 years. However,
this population appeared to be exceptional and the next largest
disparity was in a northwest Atlantic white shark underaged by 21 years
(Hamady \emph{et al.} 2014). \(\Delta_{Max}\) averaged 18 years across
all studies. For chemically marked populations the estimation of
\(\Delta_{Mean}\) and \(\Delta_{Max}\) was limited by the short duration
of studies and small sample sizes; both ranged from 2 to 3 years.

Among the 10 bomb carbon dating studies where age underestimation was
likely, 61 individual samples were pooled for analysis. With the
exception of three outliers (two northeast Pacific white sharks and one
school shark), there were consistent patterns in the relative lengths
and ages that it occurred in (Fig. 3a). Relative length, in particular,
was a strong and highly significant predictor of occurrence (Likelihood
ratio test: \(\chi^2\)=62.33, \emph{d.f.}=1, \emph{P}\textless{}0.01).
At lengths above 88\% of \(L_\infty\) more than half of all individuals
were underaged (Fig. 3b, Table 3). Relative age was also a highly
significant predictor of occurrence (Likelihood ratio test:
\(\chi^2\)=58.39, \emph{d.f.}=1, \emph{P}\textless{}0.01). At ages above
41\% of \(A_{Max}\) more than half of individuals were underaged (Fig.
3c, Table 3). This implies that, on average, ageing structures provided
a valid record of age for less than half of the maximum longevity of the
species.

Using the definitions in Table 1, the extent of age underestimation in
bomb carbon dating studies is even more pronounced, since any ages above
the age that it is first detected are effectively uncertain. Based on
these definitions, growth zones were a reliable and valid indicator of
true age for approximately a third of total lifespan (median = 34\%)
(Fig. 1, Table 2).

\subsubsection{How pervasive is age
underestimation?}\label{how-pervasive-is-age-underestimation}

Determining just how pervasive age underestimation is in sharks and rays
is difficult since chemical marking seems inherently less likely to
detect it than bomb carbon dating. It is also possible that both methods
may be unable to detect it or decouple it from other processes in
certain situations. Shark and ray age validation studies in general were
characterized by very low sample sizes -- 57\% of studies were based on
fewer than 10 samples, and 17\% were based on only a single individual
(Fig. 1, Fig. 2). The relatively high frequency of detection in spite of
these small sample sizes suggests it must be relatively common.

Age underestimation occurred in at least nine different genera (Table
2), indicating it is a widespread phenomenon. Although predominantly
detected in two orders, the Lamniformes and Carcharhiniformes, these
were also by far the most intensively studied (see Supplementary
Material). Age underestimation does not seem to be restricted to
particular life history types, and has been reported in some relatively
small and short-lived species such as blue-spotted mask rays
(\emph{Neotrygon kuhlii}, Dasyatidae) (Pierce and Bennett 2009) as well
as large and long-lived species such as dusky sharks (\emph{Carcharhinus
obscurus}, Carcharhinidae) (Natanson \emph{et al.} 2014).

Although common and widespread, it does not seem to be ubiquitous, and
was absent from some well-studied species. Age underestimation has not
been reported in gummy sharks (\emph{Mustelus antarcticus}, Triakidae) a
species studied concurrently with school sharks over many decades in
southern Australian waters (Walker \emph{et al.} 2001). Likewise, there
is no evidence for it in the well-studied leopard shark (\emph{Triakis
semifasciata}, Triakidae) (although Kusher \emph{et al.} (1992) noted
that OTC uptake was insufficient for age validation in 65\% of
recaptured sharks). There is also evidence of intraspecific variability
in its occurrence; it was absent in northwest Atlantic porbeagle sharks
(Campana \emph{et al.} 2002) but present in New Zealand porbeagles
(Francis \emph{et al.} 2007) (differences in sampling and exploitation
history (Cassoff \emph{et al.} 2007) should be noted here, however).

Two interrelated and mutually inclusive hypotheses have been proposed to
explain age underestimation in shark and ray ageing studies:

\begin{enumerate}
\def\labelenumi{\arabic{enumi})}
\item
  it results from growth zones becoming vanishingly small and
  unresolvable on the margin of ageing structures as individuals get
  larger and older, and as growth decreases (Francis \emph{et al.} 2007;
  Chin \emph{et al.} 2013; Hamady \emph{et al.} 2014), or
\item
  it results from a temporary or permanent cessation of growth and
  growth zone formation (Natanson \emph{et al.} 2014) potentially
  occurring at any life stage, including in response to a reallocation
  of resources to reproduction (Natanson 1993; Harry \emph{et al.} 2013)
  or from ``feast and famine'' feeding behavior (Andrews and Kerr 2015).
\end{enumerate}

Current evidence does not clearly favor either hypothesis. On the one
hand, age determination in many sharks and rays is inherently difficult,
and progress in ageing slow due in large part to the limitations of
available ageing structures (Cailliet and Goldman 2004; Goldman \emph{et
al.} 2012). Intraspecifically, counts and readability of growth zones
have been shown to vary depending on the method of preparation (Irvine
\emph{et al.} 2006), reader experience (Officer \emph{et al.} 1996),
type of structure used (Bubley \emph{et al.} 2012), and region of the
vertebral column sampled (Natanson and Skomal 2015). The hypothesis that
growth zones simply become unresolvable is appealing since it implies
that, with the right method of preparation, the true age may still be
attainable. However, it also risks overlooking the extensive research
done with the aim of enhancing the readability of ageing structures.
This includes histology (Natanson 1993), radiography (Aasen 1963),
scanning x-ray fluorescence microscopy (Raoult \emph{et al.} 2016),
micro-computed tomography (Geraghty \emph{et al.} 2012) and a multitude
of staining and sectioning techniques (Cailliet \emph{et al.} 1983;
Goldman \emph{et al.} 2012). While these techniques have sometimes
revealed underageing (Francis \emph{et al.} 2007; Maurer 2009), none has
yet indicated that counts may be underestimated by the large magnitudes
documented here.

On the other hand, the hypothesis that age underestimation is due to
growth zones ceasing to form also has foundation. Otolith mineralization
differs from most other calcified structures found in vertebrates,
taking place within an acellular medium known as endolymph with
deposition of new material occurring daily (Morales-Nin 2000; Payan
\emph{et al.} 2004). Francis \emph{et al.} (2007) note that ``vertebrae
form part of the axial skeleton of a shark so, in theory, if the shark's
somatic growth ceases, then deposition of material on the outer margin
of the vertebrae should also cease''. Elasmobranch vertebral centra, the
structures used most commonly for age determination, are a form of
densely calcified `areolar' cartilage found uniquely within the
vertebrae (Dean and Summers 2006). Unlike bone, mineral content in shark
and ray cartilage has been shown to vary within individuals,
intraspecifically, and interspecifically (Porter \emph{et al.} 2007).
Indeed, the vertebral centra of sharks and rays are so variable that
they were initially thought to be useful for taxonomic classification
(Ridewood 1921). In addition to this, the physiological and mechanical
factors that mediate mineralization of vertebrae are still largely
unknown (Dean and Summers 2006; Porter \emph{et al.} 2007), despite long
being recognized as centrally important to ageing (Cailliet \emph{et
al.} 1986). So while counts of growth zones obtained from shark and ray
ageing structures are essentially treated the same as those from
otoliths, the underlying processes that generate them have not been
demonstrated to be the same, and may well differ.

Evidence from bomb carbon dating, where age underestimation manifested
almost entirely in larger and older individuals, is consistent with the
suggestion that growth zone periodicity changes or ceases later in life,
potentially after the onset of sexual maturity (Casey and Natanson 1992;
Natanson \emph{et al.} 2016). While no conclusive link with maturation
has been demonstrated, this would provide a physiological mechanism to
explain age underestimation. Growth zone formation has also recently
been shown to shift from biannual to annual after maturity in shortfin
mako vertebrae (Kinney \emph{et al.} 2016). Many sharks and rays have
reproductive cycles that last two or more years, so if the timing and
frequency of growth zones are linked to reproductive events, associated
changes in the frequency of growth zone formation could be difficult to
detect, and may only gradually become evident after several full
reproductive cycles.

At present little can be definitively said about the mechanisms
underlying age underestimation. In summary though, it appears to be
common and widespread and manifests in similar ways across a range of
species. On the balance of this evidence, it seems likely to be a
systemic issue associated with the current methods and structures used
for ageing sharks and rays. While it may not occur in all species, it
will probably continue to be detected as validation studies increase and
expand across taxa, particularly if ageing methods remain unchanged.

\subsubsection{Implications}\label{implications}

The findings of this study are relevant to the numerous species that
have been and continue to be aged by counting growth zones on calcified
structures. Importantly, this study confirms that it is no longer
sufficient to assume growth zones are deposited regularly on ageing
structures throughout life, reinforcing the need for ongoing validation
of all age classes. Yet while numerous studies have now documented age
underestimation, there has still been no directed investigation of its
effects and only limited discussion of its possible implications. As a
starting point I consider the potential effects in relation to two
fundamental population processes, growth and mortality.

\paragraph{Growth}\label{growth}

One of the most clearly apparent effects of underestimating age relates
to modelling growth. Since ages from growth zone counts are an unknown
function of true age, it follows that any growth parameters estimated
from these data are likely to be biased. Francis \emph{et al.} (2007),
the only authors to so far correct for age underestimation, used a
modified Von Bertalanffy growth function to adjust growth zone counts to
reflect true ages. Natanson \emph{et al.} (2016) on the other hand,
opted to remove all ages deemed unreliable (\textgreater{} 14 years) and
used only back-calculated data when re-fitting growth curves to account
for age underestimation in common thresher sharks (\emph{Alopias
vulpinus}, Alopiidae). In both these particular cases the resultant
growth curves were similar to those fit to the original data, suggesting
that any bias was minimal. Indeed, this may be expected since age was
underestimated in individuals that had completed most of their growth
already.

Simulating data from the models in Francis \emph{et al.} (2007) suggests
that age underestimation could still have a pronounced impact,
especially when sample sizes are small. Through the underestimation of
age, there is an apparent `loss' of older individuals from length-at-age
data. This effectively truncates the asymptote of the growth curve,
presumably making it more difficult to obtain an unbiased estimate of
\(L_\infty\) (Fig. 4a). \(L_\infty\) is strongly, negatively correlated
with the growth coefficient, \emph{k}, in many commonly used models for
describing fish growth. As such, a positively biased (larger)
\(L_\infty\) associated with this apparent absence of older individuals
would lead to a negatively biased (lower) \emph{k}, somewhat analogous
to the bias generated by length-selective fishing gears such as gillnets
(Walker \emph{et al.} 1998; Thorson and Simpfendorfer 2009). This
phenomenon was seen in northwest Atlantic dusky sharks; using validated
data points corrected for age underestimation led to an increase in the
estimated growth coefficient of 6-17\% and a decrease in \(L_\infty\) of
8-9\% (Natanson \emph{et al.} 2014). Growth coefficients, \emph{k},
estimated from vertebral ageing of school sharks, where ages are known
to be substantially underestimated, were 24-26\% lower than estimated
from two long-term tagging studies (Moulton \emph{et al.} 1992). Thus, a
paradox of underestimating age is that the bias introduced may lead to
an underestimate of the growth coefficient, not an overestimate, as
would be intuitively expected.

Francis \emph{et al.} (2007) showed that simple corrections can account
for some of the bias associated with age underestimation. However, under
even slightly more complex assumptions it is clear that additional
biases would be more difficult to correct for. For example, assuming
that the growth coefficient (i.e. \emph{k}) varies among individuals in
a population and that age underestimation is predominantly a function of
length, it would be expected to disproportionately impact faster growing
individuals (Fig. 4b). At the very least it is likely to further add to
the challenge of estimating robust growth parameters in sharks and rays
by compounding biases caused by length selective fishing, low sample
sizes, and growth model uncertainty.

\paragraph{Mortality}\label{mortality}

The more directly obvious impact of age underestimation is on longevity.
The frequency and magnitude of age underestimation documented here
indicates that many sharks and rays are likely to live much longer than
currently thought. This is important since longevity is frequently used
to make inferences about natural mortality, \emph{M}, a parameter that
is highly influential in population models and that nearly always has to
be pre-specified for information-limited taxa such as sharks and rays
(Kenchington 2014). Reduced longevity is also symptomatic of a more
general `loss of age structure' caused by age underestimation, which
presents a potentially more serious issue than biased growth parameters.
When using age structured population models that are fit to these data
such as catch curves (Simpfendorfer 1999; Robbins \emph{et al.} 2006),
this apparent loss of population age structure could be inadvertently
attributed to or indistinguishable from fishing mortality (Fig. 4c).

Understanding the impacts of these biases on populations, which are
influenced by complex and non-linear dynamics, is not straightforward.
Shorter-lived species are generally more productive (i.e.~have a higher
recruitment). The much greater longevity found for many species may
indicate they have a lower natural mortality and in turn lower
productivity and resilience to fishing than currently thought. In
contrast, in a simple demographic analysis assuming Hoenig mortality for
New Zealand porbeagles, use of the true longevity of 65 years results in
a much greater net reproductive rate, \(R_0\), than if the apparent
longevity of 38 years is used. This translates to a population doubling
time of \textasciitilde{}9 years compared with \textasciitilde{}22 years
(Fig. 4d), implying much greater population productivity. In sand tiger
sharks (\emph{Carcharias taurus}, Odontaspididae) it was noted that the
upward revision of lifespan to 40 years would lead to at least five
times more reproductive years than previously thought, again implying
increased productivity (Passerotti \emph{et al.} 2014).

\begin{center}
\noindent\rule{2cm}{0.4pt}
\end{center}

These examples, while simplistic, highlight the potential ways that age
underestimation may have skewed understanding of shark and ray
population dynamics. With only a handful of long-term, empirical
observations of wild shark and ray populations (e.g.~Feldheim \emph{et
al.} (2014)), current perceptions of population dynamics, including key
areas such as fisheries productivity reference points and extinction
risk, are strongly shaped by comparative life history studies (Cortés
2000; Frisk \emph{et al.} 2001, 2005; Pardo \emph{et al.} 2016) and
meta-analyses (García \emph{et al.} 2008; Hutchings \emph{et al.} 2012;
Zhou \emph{et al.} 2012) that draw heavily upon age and growth studies.

Such are the data limitations for many populations that assessments are
often based solely on life history information (Cortés 2002, 2016;
Brooks \emph{et al.} 2009). This means that the above issues have the
potential to extend well beyond the specific examples above. For
example, age-at-maturity, another key determinant of population
productivity (Smith \emph{et al.} 1998; Pardo \emph{et al.} 2016), is
frequently obtained from the inverse of a growth curve at
length-at-maturity, meaning it too would be susceptible to biased growth
parameters. Sidestepping the use of biased parameters altogether is also
difficult. While it may be possible to use assessment techniques that
avoid specifying, say, longevity (Xiao 2002; Skalski \emph{et al.}
2008), it is generally not possible to avoid \emph{M} altogether.
Furthermore, almost all life history invariant techniques that are used
to pre-specify \emph{M} require the use of length-at-age, growth
parameters, or longevity (Kenchington 2014; Then \emph{et al.} 2014).
These dependencies mean there is considerable potential for ageing error
to propagate through population assessments.

Through these types of biases, age underestimation may have important
flow-on effects for fisheries management and conservation. With
age-structured population models increasingly being used to support
management of commercially exploited stocks (Cortés \emph{et al.} 2012),
model misspecficiation due to ageing error has the potential to affect
scientific advice such as acceptable catch limits. The impacts that
these types of errors can have are well documented in teleosts (Yule
\emph{et al.} 2008; Melvin and Campana 2010). For example,
underestimation of longevity in orange roughy (\emph{Hoplostethus
atlanticus}, Trachichthyidae) (Smith \emph{et al.} 1995) and Pacific
ocean perch (\emph{Sebastes alutus}, Sebastidae) (Beamish 1979) led to
overly optimistic estimates of stock productivity, in both cases
contributing to serious, long-term ecological and socio-economic impacts
on these resources.

Given that sharks and rays are often not the target of fishing
activities, management strategies are typically less responsive. They
are often centered around the identification and protection of high-risk
species or mitigation of high-risk activities using measures such as
spatial and temporal closures and species catch restrictions. Because of
this, the ultimate impacts of age underestimation from a conservation or
fisheries management perspective may well be more insidious, occurring
through the misclassification of high-risk species, leading to
inefficient prioritization of research, monitoring, and management
measures. If age underestimation is indeed as widespread and common as
indicated from this study, the impacts could also be substantial from a
wider scientific perspective, affecting the many disciplines that also
use baseline life history data.

The implications of age underestimation are also by no means restricted
to sharks and rays, and this study serves as a reminder of the potential
impacts on fish more generally. While the processes governing otolith
mineralization are well understood, many species, particularly
long-lived and deep-water, are nonetheless difficult to age, which can
lead to underestimation of true age. Similar phase-shifts to those
reviewed here have been documented in bocaccio rockfish (\emph{Sebastes
paucispinis}, Sebastidae) (Andrews \emph{et al.} 2005), for example. The
use of whole rather than sectioned otoliths has also frequently been
shown to underestimate age (Bennett \emph{et al.} 1982; Dwyer \emph{et
al.} 2016). For example, the use of whole otoliths likely resulted in a
biased growth curve for southern bluefin tuna (\emph{Thunnus maccoyii},
Scombridae) (Gunn \emph{et al.} 2008). In addition to this, a range of
structures other than otoliths are routinely used to age fish, many of
which have been shown to underestimate age including scales (Secor
\emph{et al.} 1995), fin-rays (Rien and Beamesderfer 1994), and
vertebrae (Gunn \emph{et al.} 2008). In most of these instances, the
impacts of ageing error are presumably isolated to specific populations.
Nonetheless, for groups of species that are difficult to or unable to be
aged using otoliths, the potential exists for more widespread biases to
manifest.

\subsubsection{Confronting the challenges of shark and ray
ageing}\label{confronting-the-challenges-of-shark-and-ray-ageing}

The outcomes of this study highlight the ongoing difficulties of ageing
sharks and rays using calcified structures, particularly the validation
of growth zones in older individuals. At present it is unclear whether
the underestimation of ages is a major problem, or perhaps a simple bias
that, for some applications, could be corrected with adjustments like
those made to otolith back-calculations (Campana 1990) or worn spines
(Taylor \emph{et al.} 2013). Although this study is the first time data
have been synthesized and quantified on age underestimation, the issue
is far from new (Kalish and Johnston 2001). But despite repeated
invalidation of ageing structures in older individuals of a range of
species, it has not yet led to widespread changes to ageing practices or
growth modeling.

Understanding the extent of the problem and ultimately addressing it
will require, in the first instance, more data. While the number of age
validation studies has increased in recent years, the studies reviewed
here cover fewer than 5\% of described species (Fig. 1 and Fig. 2). As
noted by Cailliet (2015) and emphasized by Figure 2, even fewer of these
species have been validated convincingly. Bomb carbon dating appears to
be the most effective tool currently available for age validation of
sharks and rays; it is the most likely to give a true indication of
longevity and detect whether age has been underestimated. The high cost
of such studies, the increasing difficulty of obtaining archived samples
born prior to the rise in \(\Delta^{14}\textup{C}\), and the
unsuitability for shorter-lived species mean it is not a `silver
bullet'.

Fluorochrome chemical marking is clearly an effective tool in some
situations, however this study raises questions about its usefulness for
validating older age-classes and why it seemingly fails in many cases.
The ongoing use of chemical markers in laboratory-based studies to
quantify mineralization rates may help resolve some of these questions
(Officer \emph{et al.} 1997), particularly if they can be extended to
examine the influence of factors such as length, age, and maturity
stage. Extending the inferences from these studies to wild populations
is always problematic, and therefore long-term tag-recapture studies
targeting both juveniles and adults are the only way to convincingly
validate age and the efficacy of chemical marking in older individuals.
As this study shows, the costs and logistical difficulties of conducting
either type of study have so far made them feasible for a very small
number of stocks globally, and this seems unlikely to change in the
foreseeable future.

As the enormity and complexity of the conservation challenges facing
sharks and rays globally becomes increasingly clear (Dulvy \emph{et al.}
2014, 2017), there may be a need to confront the wider issue, that
calcified structures may be insufficient to meet the ageing needs of
many species. While there are many excellent examples that illustrate
that sharks and rays can be aged successfully using calcified
structures, after more than 50 years of ageing, one of the most
consistent feature of such structures is their inconsistency (Goldman
\emph{et al.} 2012; Cailliet 2015). With no foreseeable end to the
difficulties associated with traditional ageing structures, it may be
time to look for solutions that complement existing methods and can
assist in building the deeper understanding of growth and mortality
needed for effective conservation and management.

One such solution could include a shift in focus toward the use of
alternative data sources, for example tag-recapture and length-based
data for estimating growth. Although commonly used, there has been a
tendency to view these data as less credible compared to those derived
from calcified ageing structures which, until now, have been seen as the
more reliable source of information. Indeed, many of the studies
reviewed here also included tag-recapture components that were often
discussed in less detail and rarely considered the `preferred' growth
model (Davenport and Stevens 1988; McAuley \emph{et al.} 2006; Kneebone
\emph{et al.} 2008; Andrews \emph{et al.} 2011). Comparison of growth
parameters derived from populations studied simultaneous using both
methods may also be one of the only ways to empirically determine
whether some of the expected biases proposed here really exist.

In recent years, tagging programs for sharks and rays have increasingly
been initiated for various non-fishery related applications
(e.g.~acoustic telemetry, ecological study, human-safety) and by
non-fisheries institutions (e.g.~universities, ENGOs). Such programs can
potentially be of considerable value in supplementing the limited
resources of fisheries agencies for obtaining long-term data on growth.
The use of tagging and length-based methods is also consistent with the
shift towards non-lethal alternatives to sampling that are now
increasingly being advocated (Hammerschlag and Sulikowski 2011).
Opportunities for collecting and making tag-recapture data publicly
available for age and growth estimation purposes should be embraced
where possible.

\subsection{Conclusions}\label{conclusions}

\begin{quote}
\emph{The prevalence and impact of inaccurate age determinations on the
accuracy of population dynamics studies cannot be overstated} (Campana
2001)
\end{quote}

The study of shark and ray life history has evolved in much the same way
as in teleost fish. It is, in large part, built upon the ageing of
individuals from which parameters such as growth, age at maturity,
longevity, and mortality are estimated or indirectly inferred. The
simple assumption underpinning this process is that counts of growth
zones do indeed provide a valid record of age throughout life. The
comprehensive review of bomb carbon dating and chemical marking age
validation studies undertaken here shows that this assumption is
frequently violated, highlighting a systemic problem associated with the
most common method and structures used for ageing. Given the inherent
difficulties in studying wild populations, age and growth studies form
the basis of much of the current understanding of shark and ray
population dynamics. So while age underestimation, in isolation, may be
regarded as a relatively unimportant or minor issue, its impacts could
still be far reaching and warrant closer attention. The ongoing
potential for it to compound the existing, significant challenges in
obtaining accurate estimates of growth, longevity, and mortality for
sharks and rays should not be overlooked.

\subsection{Acknowledgements}\label{acknowledgements}

This manuscript benefited greatly from detailed and insightful reviews
by L. J. Natanson, E. Cort\(\acute{\textup{e}}\)s, and a third,
anonymous reviewer. Thank you to D. S. Waltrick and L. F. N. Waltrick
for your support and encouragement.

\newpage

\subsection*{References}\label{references}
\addcontentsline{toc}{subsection}{References}

\hypertarget{refs}{}
\hypertarget{ref-aasen_length_1963}{}
Aasen, O. (1963) Length and growth of the porbeagle (\emph{Lamna nasus},
Bonnaterre) in the North West Atlantic. \emph{Fiskeridirektoratets
skrifter, Serie Havundersøkelser} \textbf{13}, 20--37.

\hypertarget{ref-andrews_validated_2015}{}
Andrews, A.H. and Kerr, L.A. (2015) Validated age estimates for large
white sharks of the northeastern Pacific Ocean: Altered perceptions of
vertebral growth shed light on complicated bomb \(\Delta\)14C results.
\emph{Environmental Biology of Fishes} \textbf{98}, 971--978.
doi:\href{https://doi.org/10.1007/s10641-014-0326-8}{10.1007/s10641-014-0326-8}.

\hypertarget{ref-andrews_bomb_2005}{}
Andrews, A.H., Burton, E.J., Kerr, L.A., Cailliet, G.M., Coale, K.H.,
Lundstrom, C.C. and Brown, T.A. (2005) Bomb radiocarbon and lead--Radium
disequilibria in otoliths of bocaccio rockfish (\emph{Sebastes
paucispinis}): A determination of age and longevity for a
difficult-to-age fish. \emph{Marine and Freshwater Research}
\textbf{56}, 517--528.
doi:\href{https://doi.org/10.1071/MF04224}{10.1071/MF04224}.

\hypertarget{ref-andrews_bomb_2011}{}
Andrews, A.H., Natanson, L.J., Kerr, L.A., Burgess, G.H. and Cailliet,
G.M. (2011) Bomb radiocarbon and tag-recapture dating of sandbar shark
(\emph{Carcharhinus plumbeus}). \emph{Fishery Bulletin} \textbf{109},
454--465.

\hypertarget{ref-ardizzone_application_2006}{}
Ardizzone, D., Cailliet, G.M., Natanson, L.J., Andrews, A.H., Kerr, L.A.
and Brown, T.A. (2006) Application of bomb radiocarbon chronologies to
shortfin mako (\emph{Isurus oxyrinchus}) age validation.
\emph{Environmental Biology of Fishes} \textbf{77}, 355--366.
doi:\href{https://doi.org/10.1007/s10641-006-9106-4}{10.1007/s10641-006-9106-4}.

\hypertarget{ref-beamish_new_1979}{}
Beamish, R.J. (1979) New Information on the longevity of Pacific Ocean
Perch (\emph{Sebastes alutus}). \emph{Journal of the Fisheries Research
Board of Canada} \textbf{36}, 1395--1400.
doi:\href{https://doi.org/10.1139/f79-199}{10.1139/f79-199}.

\hypertarget{ref-beamish_forgotten_1983}{}
Beamish, R.J. and Mcfarlane, G.A. (1983) The forgotten requirement for
age validation in fisheries biology. \emph{Transactions of the American
Fisheries Society} \textbf{112}, 735--743.
doi:\href{https://doi.org/10.1577/1548-8659(1983)112\%3C735:TFRFAV\%3E2.0.CO;2}{10.1577/1548-8659(1983)112\textless{}735:TFRFAV\textgreater{}2.0.CO;2}.

\hypertarget{ref-bennett_confirmation_1982}{}
Bennett, J.T., Boehlert, G.W. and Turekian, K.K. (1982) Confirmation of
longevity in \emph{Sebastes diploproa} (Pisces: Scorpaenidae) from
210Pb/226Ra measurements in otoliths. \emph{Marine Biology} \textbf{71},
209--215.
doi:\href{https://doi.org/10.1007/BF00394632}{10.1007/BF00394632}.

\hypertarget{ref-brooks_analytical_2009}{}
Brooks, E.N., Powers, J.E. and Cortés, E. (2009) Analytical reference
points for age-structured models: Application to data-poor fisheries.
\emph{ICES Journal of Marine Science: Journal du Conseil}, fsp225.
doi:\href{https://doi.org/10.1093/icesjms/fsp225}{10.1093/icesjms/fsp225}.

\hypertarget{ref-bubley_reassessment_2012}{}
Bubley, W.J., Kneebone, J., Sulikowski, J.A. and Tsang, P.C.W. (2012)
Reassessment of spiny dogfish \emph{Squalus acanthias} age and growth
using vertebrae and dorsal-fin spines. \emph{Journal of Fish Biology}
\textbf{80}, 1300--1319.
doi:\href{https://doi.org/10.1111/j.1095-8649.2011.03171.x}{10.1111/j.1095-8649.2011.03171.x}.

\hypertarget{ref-cailliet_perspectives_2015}{}
Cailliet, G. (2015) Perspectives on elasmobranch life-history studies: A
focus on age validation and relevance to fishery management.
\emph{Journal of fish biology} \textbf{87}, 1271--1292.
doi:\href{https://doi.org/10.1111/jfb.12829}{10.1111/jfb.12829}.

\hypertarget{ref-cailliet_elasmobranch_1990}{}
Cailliet, G.M. (1990) Elasmobranch age determination and verification:
An updated review. In: \emph{Elasmobranchs as living resources: Advances
in the biology, ecology, systematics, and the status of the fisheries
NOAA Technical Report NMFS 90}. (eds H.L. Pratt, S.H. Gruber and T.
Taniuchi). pp 157--165.

\hypertarget{ref-cailliet_age_2004}{}
Cailliet, G.M. and Goldman, K.J. (2004) Age determination and validation
in chondrichthyan fishes. In: \emph{Biology of sharks and their
relatives}. (eds J.C. Carrier, J.A. Musick and M.R. Heithaus). CRC
Press, Boca Raton, pp 399--448.

\hypertarget{ref-cailliet_techniques_1983}{}
Cailliet, G.M., Martin, L.K., Kusher, D., Wolf, P. and Welden, B.A.
(1983) Techniques for enhancing vertebral bands in age estimation of
California elasmobranchs. In: \emph{Proceedings of the International
Workshop on Age Determination of Oceanic Pelagic Fishes: Tunas,
Billfishes, and Sharks. 1982. NOAA Technical Report NMFS 90}. (eds E.D.
Prince and L.M. Pulos). US Department of Commerce, National Oceanic and
Atmospheric Administration, National Marine Fisheries Service,
Scientific Publications Office, National Marine Fisheries Service, NOAA,
pp 157--165.

\hypertarget{ref-cailliet_elasmobranch_1986}{}
Cailliet, G.M., Radtke, R.L. and Welden, B.A. (1986) Elasmobranch age
determination and verification: A review. In: \emph{Indo-Pacific Fish
Biology: Proceedings of the second international conference on
Indo-Pacific fishes}. (eds R. Uyeno, T. Arai, T. Taniuchi and K.
Matsuura). Ichthyological Society of Japan, Tokyo, pp 345--360.

\hypertarget{ref-cailliet_age_2006}{}
Cailliet, G.M., Smith, W.D., Mollet, H.F. and Goldman, K.J. (2006) Age
and growth studies of chondrichthyan fishes: The need for consistency in
terminology, verification, validation, and growth function fitting.
\emph{Environmental Biology of Fishes} \textbf{77}, 211--228.
doi:\href{https://doi.org/10.1007/978-1-4020-5570-6_2}{10.1007/978-1-4020-5570-6\_2}.

\hypertarget{ref-campana_accuracy_2001}{}
Campana, S.E. (2001) Accuracy, precision and quality control in age
determination, including a review of the use and abuse of age validation
methods. \emph{Journal of Fish Biology} \textbf{59}, 197--242.
doi:\href{https://doi.org/10.1006/jfbi.2001.1668}{10.1006/jfbi.2001.1668}.

\hypertarget{ref-campana_chemistry_1999}{}
Campana, S.E. (1999) Chemistry and composition of fish otoliths:
Pathways, mechanisms and applications. \emph{Marine Ecology Progress
Series} \textbf{188}, 263--297.
doi:\href{https://doi.org/10.3354/meps188263}{10.3354/meps188263}.

\hypertarget{ref-campana_how_1990}{}
Campana, S.E. (1990) How reliable are growth back-calculations based on
otoliths? \emph{Canadian Journal of Fisheries and Aquatic Sciences}
\textbf{47}, 2219--2227.
doi:\href{https://doi.org/10.1139/f90-246}{10.1139/f90-246}.

\hypertarget{ref-campana_bomb_2006}{}
Campana, S.E., Jones, C., McFarlane, G.A. and Myklevoll, S. (2006) Bomb
dating and age validation using the spines of spiny dogfish
(\emph{Squalus acanthias}). \emph{Environmental Biology of Fishes}
\textbf{77}, 327--336.
doi:\href{https://doi.org/10.1007/s10641-006-9107-3}{10.1007/s10641-006-9107-3}.

\hypertarget{ref-campana_bomb_2002}{}
Campana, S.E., Natanson, L.J. and Myklevoll, S. (2002) Bomb dating and
age determination of large pelagic sharks. \emph{Canadian Journal of
Fisheries and Aquatic Sciences} \textbf{59}, 450--455.
doi:\href{https://doi.org/10.1139/f02-027}{10.1139/f02-027}.

\hypertarget{ref-casey_revised_1992}{}
Casey, J.G. and Natanson, L.J. (1992) Revised estimates of age and
growth of the sandbar shark (\emph{Carcharhinus plumbeus}) from the
western North-Atlantic. \emph{Canadian Journal of Fisheries and Aquatic
Sciences} \textbf{49}, 1474--1477.
doi:\href{https://doi.org/10.1139/f92-162}{10.1139/f92-162}.

\hypertarget{ref-cassoff_changes_2007}{}
Cassoff, R.M., Campana, S.E. and Myklevoll, S. (2007) Changes in
baseline growth and maturation parameters of Northwest Atlantic
porbeagle, \emph{Lamna nasus}, following heavy exploitation.
\emph{Canadian Journal of Fisheries and Aquatic Sciences} \textbf{64},
19--29. doi:\href{https://doi.org/10.1139/F06-167}{10.1139/F06-167}.

\hypertarget{ref-chin_validated_2013}{}
Chin, A., Simpfendorfer, C., Tobin, A. and Heupel, M. (2013) Validated
age, growth and reproductive biology of \emph{Carcharhinus
melanopterus}, a widely distributed and exploited reef shark.
\emph{Marine and Freshwater Research} \textbf{64}, 965--975.
doi:\href{https://doi.org/10.1071/MF13017}{10.1071/MF13017}.

\hypertarget{ref-cortes_incorporating_2002}{}
Cortés, E. (2002) Incorporating uncertainty into demographic modeling:
Application to shark populations and their conservation.
\emph{Conservation Biology} \textbf{16}, 1048--1062.
doi:\href{https://doi.org/10.1046/j.1523-1739.2002.00423.x}{10.1046/j.1523-1739.2002.00423.x}.

\hypertarget{ref-cortes_life_2000}{}
Cortés, E. (2000) Life history patterns and correlations in sharks.
\emph{Reviews in Fisheries Science} \textbf{8}, 299--344.
doi:\href{https://doi.org/10.1080/10408340308951115}{10.1080/10408340308951115}.

\hypertarget{ref-cortes_perspectives_2016}{}
Cortés, E. (2016) Perspectives on the intrinsic rate of population
growth. \emph{Methods in Ecology and Evolution} \textbf{7}, 1136--1145.
doi:\href{https://doi.org/10.1111/2041-210X.12592}{10.1111/2041-210X.12592}.

\hypertarget{ref-cortes_population_2012}{}
Cortés, E., Brooks, E.N. and Gedamke, T. (2012) Population dynamics,
demography, and stock assessment. In: \emph{Biology of Sharks and Their
Relatives, Second Edition}. (eds J.C. Carrier, J.A. Musick and M.R.
Heithaus). CRC Press, New York, pp 453--86.

\hypertarget{ref-davenport_age_1988}{}
Davenport, S. and Stevens, J.D. (1988) Age and growth of two
commercially important sharks (\emph{Carcharhinus tilstoni} and \emph{C.
sorrah}) from Northern Australia. \emph{Australian Journal of Marine and
Freshwater Research} \textbf{39}, 417--433.
doi:\href{https://doi.org/10.1071/MF9880417}{10.1071/MF9880417}.

\hypertarget{ref-dean_mineralized_2006}{}
Dean, M. and Summers, A. (2006) Mineralized cartilage in the skeleton of
cartilaginous fishes. \emph{Zoology} \textbf{109}, 164--168.
doi:\href{https://doi.org/10.1016/j.zool.2006.03.002}{10.1016/j.zool.2006.03.002}.

\hypertarget{ref-dulvy_methods_2004}{}
Dulvy, N.K., Ellis, J.R., Goodwin, N.B., Grant, A., Reynolds, J.D. and
Jennings, S. (2004) Methods of assessing extinction risk in marine
fishes. \emph{Fish and Fisheries} \textbf{5}, 255--276.
doi:\href{https://doi.org/10.1111/j.1467-2679.2004.00158.x}{10.1111/j.1467-2679.2004.00158.x}.

\hypertarget{ref-dulvy_extinction_2014}{}
Dulvy, N.K., Fowler, S.L., Musick, J.A., Cavanagh, R.D., Kyne, P.M.,
Harrison, L.R., Carlson, J.K., Davidson, L.N., Fordham, S.V., Francis,
M.P., Pollock, C.M., Simpfendorfer, C.A., Burgess, G.H., Carpenter,
K.E., Compagno, L.J., Ebert, D.A., Gibson, C., Heupel, M.R.,
Livingstone, S.R., Sanciangco, J.C., Stevens, J.D., Valenti, S. and
White, W.T. (2014) Extinction risk and conservation of the world's
sharks and rays. \emph{eLife} \textbf{3}, e00590.
doi:\href{https://doi.org/10.7554/eLife.00590}{10.7554/eLife.00590}.

\hypertarget{ref-dulvy_challenges_2017}{}
Dulvy, N.K., Simpfendorfer, C.A., Davidson, L.N., Fordham, S.V.,
Bräutigam, A., Sant, G. and Welch, D.J. (2017) Challenges and priorities
in shark and ray conservation. \emph{Current Biology} \textbf{27},
R565--R572.
doi:\href{https://doi.org/10.1016/j.cub.2017.04.038}{10.1016/j.cub.2017.04.038}.

\hypertarget{ref-dwyer_age_2016}{}
Dwyer, K.S., Treble, M.A. and Campana, S.E. (2016) Age and growth of
Greenland Halibut (\emph{Reinhardtius hippoglossoides}) in the Northwest
Atlantic: A changing perception based on bomb radiocarbon analyses.
\emph{Fisheries Research} \textbf{179}, 342--350.
doi:\href{https://doi.org/10.1016/j.fishres.2016.01.016}{10.1016/j.fishres.2016.01.016}.

\hypertarget{ref-feldheim_two_2014}{}
Feldheim, K.A., Gruber, S.H., DiBattista, J.D., Babcock, E.A., Kessel,
S.T., Hendry, A.P., Pikitch, E.K., Ashley, M.V. and Chapman, D.D. (2014)
Two decades of genetic profiling yields first evidence of natal
philopatry and long-term fidelity to parturition sites in sharks.
\emph{Molecular Ecology} \textbf{23}, 110--117.
doi:\href{https://doi.org/10.1111/mec.12583}{10.1111/mec.12583}.

\hypertarget{ref-fischer_review_2012}{}
Fischer, J., Erikstein, K., D'Offay, B., Guggisberg, S. and Barone, M.
(2012) Review of the implementation of the International Plan of Action
for the Conservation and Management of Sharks. No. 1076, 120 pp. FAO,
Rome.

\hypertarget{ref-francis_age_2007}{}
Francis, M.P., Campana, S.E. and Jones, C.M. (2007) Age under-estimation
in New Zealand porbeagle sharks (\emph{Lamna nasus}): Is there an upper
limit to ages that can be determined from shark vertebrae? \emph{Marine
and Freshwater Research} \textbf{58}, 10--23.
doi:\href{https://doi.org/10.1071/MF06069}{10.1071/MF06069}.

\hypertarget{ref-frazier_validated_2014}{}
Frazier, B., Driggers, W., Adams, D., Jones, C. and Loefer, J. (2014)
Validated age, growth and maturity of the bonnethead \emph{Sphyrna
tiburo} in the western North Atlantic Ocean. \emph{Journal of fish
biology} \textbf{85}, 688--712.
doi:\href{https://doi.org/10.1111/jfb.12450}{10.1111/jfb.12450}.

\hypertarget{ref-frisk_life_2005}{}
Frisk, M.G., Miller, T.J. and Dulvy, N.K. (2005) Life histories and
vulnerability to exploitation of elasmobranchs: Inferences from
elasticity, pertubation and phylogenetic analyses. \emph{Journal of
Northwest Atlantic Fishery Science} \textbf{35}, 27--45.

\hypertarget{ref-frisk_estimation_2001}{}
Frisk, M.G., Miller, T.J. and Fogarty, M.J. (2001) Estimation and
analysis of biological parameters in elasmobranch fishes: A comparative
life history study. \emph{Canadian Journal of Fisheries and Aquatic
Sciences} \textbf{58}, 969--981.
doi:\href{https://doi.org/10.1139/f01-051}{10.1139/f01-051}.

\hypertarget{ref-garcia_importance_2008}{}
García, V.B., Lucifora, L.O. and Myers, R.A. (2008) The importance of
habitat and life history to extinction risk in sharks, skates, rays and
chimaeras. \emph{Proceedings of the Royal Society B: Biological
Sciences} \textbf{275}, 83--89.
doi:\href{https://doi.org/10.1098/rspb.2007.1295}{10.1098/rspb.2007.1295}.

\hypertarget{ref-geraghty_micro-computed_2012}{}
Geraghty, P., Jones, A., Stewart, J. and Macbeth, W. (2012)
Micro-computed tomography: An alternative method for shark ageing.
\emph{Journal of fish biology} \textbf{80}, 1292--1299.
doi:\href{https://doi.org/10.1111/j.1095-8649.2011.03188.x}{10.1111/j.1095-8649.2011.03188.x}.

\hypertarget{ref-goldman_assessing_2012}{}
Goldman, K.J., Cailliet, G.M., Andrews, A.H. and Natanson, L.J. (2012)
Assessing the age and growth of chondrichthyan fishes. In: \emph{Biology
of Sharks and Their Relatives, Second Edition}. (eds J.C. Carrier, J.A.
Musick and M.R. Heithaus). CRC Press, New York, pp 423--451.

\hypertarget{ref-gunn_age_2008}{}
Gunn, J.S., Clear, N.P., Carter, T.I., Rees, A.J., Stanley, C.A.,
Farley, J.H. and Kalish, J.M. (2008) Age and growth in southern bluefin
tuna, \emph{Thunnus maccoyii} (Castelnau): Direct estimation from
otoliths, scales and vertebrae. \emph{Fisheries Research} \textbf{92},
207--220.
doi:\href{https://doi.org/10.1016/j.fishres.2008.01.018}{10.1016/j.fishres.2008.01.018}.

\hypertarget{ref-hamady_vertebral_2014}{}
Hamady, L.L., Natanson, L.J., Skomal, G.B. and Thorrold, S.R. (2014)
Vertebral bomb radiocarbon suggests extreme longevity in white sharks.
\emph{PLoS One} \textbf{9}, e84006.
doi:\href{https://doi.org/10.1371/journal.pone.0084006}{10.1371/journal.pone.0084006}.

\hypertarget{ref-hammerschlag_killing_2011}{}
Hammerschlag, N. and Sulikowski, J. (2011) Killing for conservation: The
need for alternatives to lethal sampling of apex predatory sharks.
\emph{Endangered Species Research} \textbf{14}, 135--140.

\hypertarget{ref-harry_age_2013}{}
Harry, A.V., Tobin, A.J. and Simpfendorfer, C.A. (2013) Age, growth and
reproductive biology of the spot-tail shark, \emph{Carcharhinus sorrah},
and the Australian blacktip shark, \emph{Carcharhinus tilstoni}, from
the Great Barrier Reef World Heritage Area, north-eastern Australia.
\emph{Marine and Freshwater Research} \textbf{64}, 277--293.
doi:\href{https://doi.org/10.1071/MF12142}{10.1071/MF12142}.

\hypertarget{ref-hobday_ecological_2011}{}
Hobday, A.J., Smith, A.D.M., Stobutzki, I.C., Bulman, C., Daley, R.,
Dambacher, J.M., Deng, R.A., Dowdney, J., Fuller, M., Furlani, D.,
Griffiths, S.P., Johnson, D., Kenyon, R., Knuckey, I.A., Ling, S.D.,
Pitcher, R., Sainsbury, K.J., Sporcic, M., Smith, T., Turnbull, C.,
Walker, T.I., Wayte, S.E., Webb, H., Williams, A., Wise, B.S. and Zhou,
S. (2011) Ecological risk assessment for the effects of fishing.
\emph{Fisheries Research} \textbf{108}, 372--384.
doi:\href{https://doi.org/10.1016/j.fishres.2011.01.013}{10.1016/j.fishres.2011.01.013}.

\hypertarget{ref-holden_age_1973}{}
Holden, M. and Vince, M. (1973) Age validation studies on the centra of
\emph{Raja clavata} using tetracycline. \emph{Journal du Conseil}
\textbf{35}, 13--17.
doi:\href{https://doi.org/10.1093/icesjms/35.1.13}{10.1093/icesjms/35.1.13}.

\hypertarget{ref-hutchings_life-history_2012}{}
Hutchings, J.A., Myers, R.A., García, V.B., Lucifora, L.O. and
Kuparinen, A. (2012) Life-history correlates of extinction risk and
recovery potential. \emph{Ecological Applications} \textbf{22},
1061--1067.
doi:\href{https://doi.org/10.1890/11-1313.1}{10.1890/11-1313.1}.

\hypertarget{ref-irvine_surface_2006}{}
Irvine, S.B., Stevens, J.D. and Laurenson, L.J. (2006) Surface bands on
deepwater squalid dorsal-fin spines: An alternative method for ageing
\emph{Centroselachus crepidater}. \emph{Canadian Journal of Fisheries
and Aquatic Sciences} \textbf{63}, 617--627.
doi:\href{https://doi.org/10.1139/f05-237}{10.1139/f05-237}.

\hypertarget{ref-izzo_incorporation_2007}{}
Izzo, C., Rodda, K. and Bolton, T. (2007) Incorporation time of
oxytetracycline into calcified structures of the elasmobranch
\emph{Heterodontus portusjacksoni}. \emph{Journal of Fish Biology}
\textbf{71}, 1208--1214.
doi:\href{https://doi.org/10.1111/j.1095-8649.2007.01574.x}{10.1111/j.1095-8649.2007.01574.x}.

\hypertarget{ref-kalish_determination_2001}{}
Kalish, J. and Johnston, J. (2001) Determination of school shark age
based on analysis of radiocarbon in vertebral collagen. In: \emph{Use of
the bomb radiocarbon chronometer to validate fish age. Final Report.
FRDC Project 93/109}. pp 116--129.

\hypertarget{ref-kalish_pre-and_1993}{}
Kalish, J.M. (1993) Pre-and post-bomb radiocarbon in fish otoliths.
\emph{Earth and Planetary Science Letters} \textbf{114}, 549--554.
doi:\href{https://doi.org/10.1016/0012-821X(93)90082-K}{10.1016/0012-821X(93)90082-K}.

\hypertarget{ref-kenchington_natural_2014}{}
Kenchington, T.J. (2014) Natural mortality estimators for
information-limited fisheries. \emph{Fish and Fisheries} \textbf{15},
533--562.
doi:\href{https://doi.org/10.1111/faf.12027}{10.1111/faf.12027}.

\hypertarget{ref-kerr_investigations_2006}{}
Kerr, L.A., Andrews, A.H., Cailliet, G.M., Brown, T.A. and Coale, K.H.
(2006) Investigations of \(\Delta\) 14C, \(\delta\) 13C, and \(\delta\)
15N in vertebrae of white shark (\emph{Carcharodon carcharias}) from the
eastern North Pacific Ocean. \emph{Environmental Biology of Fishes}
\textbf{77}, 337--353.
doi:\href{https://doi.org/10.1007/s10641-006-9125-1}{10.1007/s10641-006-9125-1}.

\hypertarget{ref-kinney_oxytetracycline_2016}{}
Kinney, M., Wells, R. and Kohin, S. (2016) Oxytetracycline age
validation of an adult shortfin mako shark \emph{Isurus oxyrinchus}
after 6 years at liberty. \emph{Journal of Fish Biology} \textbf{89},
1828--1833.
doi:\href{https://doi.org/10.1111/jfb.13044}{10.1111/jfb.13044}.

\hypertarget{ref-kneebone_using_2008}{}
Kneebone, J., Natanson, L., Andrews, A. and Howell, W. (2008) Using bomb
radiocarbon analyses to validate age and growth estimates for the tiger
shark, \emph{Galeocerdo cuvier}, in the western North Atlantic.
\emph{Marine Biology} \textbf{154}, 423--434.
doi:\href{https://doi.org/10.1007/s00227-008-0934-y}{10.1007/s00227-008-0934-y}.

\hypertarget{ref-kusher_validated_1992}{}
Kusher, D.I., Smith, S.E. and Cailliet, G.M. (1992) Validated age and
growth of the leopard shark, \emph{Triakis semifasciata}, with comments
on reproduction. \emph{Environmental Biology of Fishes} \textbf{35},
187--203.
doi:\href{https://doi.org/10.1007/BF00002193}{10.1007/BF00002193}.

\hypertarget{ref-maunder_review_2013}{}
Maunder, M.N. and Punt, A.E. (2013) A review of integrated analysis in
fisheries stock assessment. \emph{Fisheries Research} \textbf{142},
61--74.
doi:\href{https://doi.org/10.1016/j.fishres.2012.07.025}{10.1016/j.fishres.2012.07.025}.

\hypertarget{ref-maurer_life_2009}{}
Maurer, J. (2009) Life history of two Bering Sea slope skates:
\emph{Bathyraja lindbergi} and \emph{B. maculata}. Masters Thesis.
California State University, Montery Bay.

\hypertarget{ref-mcauley_validated_2006}{}
McAuley, R.B., Simpfendorfer, C.A., Hyndes, G.A., Allison, R.R.,
Chidlow, J.A., Newman, S.J. and Lenanton, R.C.J. (2006) Validated age
and growth of the sandbar shark, \emph{Carcharhinus plumbeus} (Nardo
1827) in the waters off Western Australia. \emph{Environmental Biology
of Fishes} \textbf{77}, 385--400.
doi:\href{https://doi.org/10.1007/s10641-006-9126-0}{10.1007/s10641-006-9126-0}.

\hypertarget{ref-mcfarlane_validation_1987}{}
McFarlane, G.A. and Beamish, R.J. (1987) Validation of the dorsal spine
method of age determination for spiny dogfish. \emph{Age and Growth of
Fish. Iowa State University Press, Ames, Iowa}, 287--300.

\hypertarget{ref-mcphie_bomb_2009}{}
McPhie, R.P. and Campana, S.E. (2009) Bomb dating and age determination
of skates (family Rajidae) off the eastern coast of Canada. \emph{ICES
Journal of Marine Science} \textbf{66}, 546--560.
doi:\href{https://doi.org/10.1093/icesjms/fsp002}{10.1093/icesjms/fsp002}.

\hypertarget{ref-melvin_high_2010}{}
Melvin, G.D. and Campana, S.E. (2010) High resolution bomb dating for
testing the accuracy of age interpretations for a short-lived pelagic
fish, the Atlantic herring. \emph{Environmental biology of fishes}
\textbf{89}, 297--311.
doi:\href{https://doi.org/10.1007/s10641-010-9663-4}{10.1007/s10641-010-9663-4}.

\hypertarget{ref-morales-nin_review_2000}{}
Morales-Nin, B. (2000) Review of the growth regulation processes of
otolith daily increment formation. \emph{Fisheries Research}
\textbf{46}, 53--67.
doi:\href{https://doi.org/10.1016/S0165-7836(00)00133-8}{10.1016/S0165-7836(00)00133-8}.

\hypertarget{ref-moulton_age_1992}{}
Moulton, P.L., Walker, T.I. and Saddlier, S.R. (1992) Age and
growth-studies of gummy shark, \emph{Mustelus antarcticus} Gunther, and
school shark, \emph{Galeorhinus galeus} (Linnaeus), from Southern
Australian waters. \emph{Australian Journal of Marine and Freshwater
Research} \textbf{43}, 1241--1267.
doi:\href{https://doi.org/10.1071/MF9921241}{10.1071/MF9921241}.

\hypertarget{ref-natanson_effect_1993}{}
Natanson, L.J. (1993) Effect of temperature on band deposition in the
little skate, \emph{Raja erinacea}. \emph{Copeia} \textbf{1}, 199--206.
doi:\href{https://doi.org/10.2307/1446311}{10.2307/1446311}.

\hypertarget{ref-natanson_age_2015}{}
Natanson, L.J. and Skomal, G.B. (2015) Age and growth of the white
shark, \emph{Carcharodon carcharias}, in the western North Atlantic
Ocean. \emph{Marine and Freshwater Research} \textbf{66}, 387--398.
doi:\href{https://doi.org/10.1071/MF14127}{10.1071/MF14127}.

\hypertarget{ref-natanson_validated_2014}{}
Natanson, L.J., Gervelis, B.J., Winton, M.V., Hamady, L.L., Gulak, S.J.
and Carlson, J.K. (2014) Validated age and growth estimates for
\emph{Carcharhinus obscurus} in the northwestern Atlantic Ocean, with
pre-and post management growth comparisons. \emph{Environmental Biology
of Fishes} \textbf{97}, 881--896.
doi:\href{https://doi.org/10.1007/s10641-013-0189-4}{10.1007/s10641-013-0189-4}.

\hypertarget{ref-natanson_analysis_2016}{}
Natanson, L.J., Hamady, L.L. and Gervelis, B.J. (2016) Analysis of bomb
radiocarbon data for common thresher sharks, \emph{Alopias vulpinus}, in
the northwestern Atlantic Ocean with revised growth curves.
\emph{Environmental Biology of Fishes} \textbf{99}, 39--47.
doi:\href{https://doi.org/10.1007/s10641-015-0452-y}{10.1007/s10641-015-0452-y}.

\hypertarget{ref-natanson_validated_2006}{}
Natanson, L.J., Kohler, N.E., Ardizzone, D., Cailliet, G.M., Wintner,
S.P. and Mollet, H.F. (2006) Validated age and growth estimates for the
shortfin mako, \emph{Isurus oxyrinchus}, in the North Atlantic Ocean.
\emph{Environmental Biology of Fishes} \textbf{77}, 367--383.
doi:\href{https://doi.org/10.1007/s10641-006-9127-z}{10.1007/s10641-006-9127-z}.

\hypertarget{ref-officer_captive_1997}{}
Officer, R.A., Day, R.W., Clement, J.G. and Brown, L.P. (1997) Captive
gummy sharks, \emph{Mustelus antarcticus}, form hypermineralised bands
in their vertebrae during winter. \emph{Canadian Journal of Fisheries
and Aquatic Sciences} \textbf{54}, 2677--2683.
doi:\href{https://doi.org/10.1139/f97-157}{10.1139/f97-157}.

\hypertarget{ref-officer_sources_1996}{}
Officer, R.A., Gason, A.S., Walker, T.I. and Clement, J.G. (1996)
Sources of variation in counts of growth increments in vertebrae from
gummy shark,(\emph{Mustelus antarcticus}, and school shark,
\emph{Galeorhinus galeus}): Implications for age determination.
\emph{Canadian Journal of Fisheries and Aquatic Sciences} \textbf{53},
1765--1777. doi:\href{https://doi.org/10.1139/f96-103}{10.1139/f96-103}.

\hypertarget{ref-pardo_maximum_2016}{}
Pardo, S.A., Kindsvater, H.K., Reynolds, J.D. and Dulvy, N.K. (2016)
Maximum intrinsic rate of population increase in sharks, rays, and
chimaeras: The importance of survival to maturity. \emph{Canadian
Journal of Fisheries and Aquatic Sciences} \textbf{73}, 1159--1163.
doi:\href{https://doi.org/10.1139/cjfas-2016-0069}{10.1139/cjfas-2016-0069}.

\hypertarget{ref-passerotti_maximum_2014}{}
Passerotti, M., Andrews, A., Carlson, J., Wintner, S., Goldman, K. and
Natanson, L. (2014) Maximum age and missing time in the vertebrae of
sand tiger shark (\emph{Carcharias taurus}): Validated lifespan from
bomb radiocarbon dating in the western North Atlantic and southwestern
Indian Oceans. \emph{Marine and Freshwater Research} \textbf{65},
674--687. doi:\href{https://doi.org/10.1071/MF13214}{10.1071/MF13214}.

\hypertarget{ref-patrick_using_2010}{}
Patrick, W.S., Spencer, P., Link, J., Cope, J., Field, J., Kobayashi,
D., Lawson, P., Gedamke, T., Cortés, E., Ormseth, O. and others (2010)
Using productivity and susceptibility indices to assess the
vulnerability of United States fish stocks to overfishing. \emph{Fishery
Bulletin} \textbf{108}, 305--322.

\hypertarget{ref-payan_endolymph_2004}{}
Payan, P., De Pontual, H., Bœuf, G. and Mayer-Gostan, N. (2004)
Endolymph chemistry and otolith growth in fish. \emph{Comptes Rendus
Palevol} \textbf{3}, 535--547.
doi:\href{https://doi.org/10.1016/j.crpv.2004.07.013}{10.1016/j.crpv.2004.07.013}.

\hypertarget{ref-pierce_validated_2009}{}
Pierce, S.J. and Bennett, M.B. (2009) Validated annual band-pair
periodicity and growth parameters of blue-spotted maskray
\emph{Neotrygon kuhlii} from south-east Queensland, Australia.
\emph{Journal of Fish Biology} \textbf{75}, 2490--2508.
doi:\href{https://doi.org/10.1111/j.1095-8649.2009.02435.x}{10.1111/j.1095-8649.2009.02435.x}.

\hypertarget{ref-porter_contribution_2007}{}
Porter, M.E., Koob, T.J. and Summers, A.P. (2007) The contribution of
mineral to the material properties of vertebral cartilage from the
smooth-hound shark \emph{Mustelus californicus}. \emph{Journal of
Experimental Biology} \textbf{210}, 3319--3327.
doi:\href{https://doi.org/10.1242/jeb.006189}{10.1242/jeb.006189}.

\hypertarget{ref-prince_proceedings_1983}{}
Prince, E. and Pulos, L. (1983) Proceedings of the International
Workshop on Age Determination of Oceanic Pelagic Fishes: Tunas,
Billfishes, and Sharks. 1982. 211 pp. NOAA Technical Report NMFS 8.

\hypertarget{ref-r_core_team_r:_2016}{}
R Core Team (2016) \emph{R: A Language and Environment for Statistical
Computing}. R Foundation for Statistical Computing, Vienna, Austria.

\hypertarget{ref-raoult_strontium_2016}{}
Raoult, V., Peddemors, V.M., Zahra, D., Howell, N., Howard, D.L., De
Jonge, M.D. and Williamson, J.E. (2016) Strontium mineralization of
shark vertebrae. \emph{Scientific Reports} \textbf{6}.
doi:\href{https://doi.org/10.1038/srep29698}{10.1038/srep29698}.

\hypertarget{ref-ridewood_calcification_1921}{}
Ridewood, W.G. (1921) On the calcification of the vertebral centra in
sharks and rays. \emph{Philosophical Transactions of the Royal Society
of London. Series B, Containing Papers of a Biological Character}
\textbf{210}, 311--407.

\hypertarget{ref-rien_accuracy_1994}{}
Rien, T.A. and Beamesderfer, R.C. (1994) Accuracy and precision of white
sturgeon age estimates from pectoral fin rays. \emph{Transactions of the
American Fisheries Society} \textbf{123}, 255--265.
doi:\href{https://doi.org/10.1577/1548-8659(1994)123\%3C0255:AAPOWS\%3E2.3.CO;2}{10.1577/1548-8659(1994)123\textless{}0255:AAPOWS\textgreater{}2.3.CO;2}.

\hypertarget{ref-robbins_ongoing_2006}{}
Robbins, W.D., Hisano, M., Connolly, S.R. and Choat, J.H. (2006) Ongoing
collapse of coral-reef shark populations. \emph{Current Biology}
\textbf{16}, 2314--2319.
doi:\href{https://doi.org/10.1016/j.cub.2006.09.044}{10.1016/j.cub.2006.09.044}.

\hypertarget{ref-secor_validation_1995}{}
Secor, D.H., Trice, T. and Hornick, H. (1995) Validation of
otolith-based ageing and a comparison of otolith and scale-based ageing
in mark-recaptured Chesapeake Bay striped bass, \emph{Morone saxatilis}.
\emph{Fishery Bulletin} \textbf{93}, 186--190.

\hypertarget{ref-simpfendorfer_mortality_1999}{}
Simpfendorfer, C.A. (1999) Mortality estimates and demographic analysis
for the Australian sharpnose shark, \emph{Rhizoprionodon taylori}, from
northern Australia. \emph{Fishery Bulletin} \textbf{97}, 978--986.

\hypertarget{ref-skalski_effects_2008}{}
Skalski, J.R., Millspaugh, J.J. and Ryding, K.E. (2008) Effects of
asymptotic and maximum age estimates on calculated rates of population
change. \emph{Ecological Modelling} \textbf{212}, 528--535.
doi:\href{https://doi.org/10.1016/j.ecolmodel.2007.11.012}{10.1016/j.ecolmodel.2007.11.012}.

\hypertarget{ref-smith_age_1995}{}
Smith, D.C., Robertson, S.G., Fenton, G.E. and Short, S.A. (1995) Age
determination and growth of orange roughy (\emph{Hoplostethus
atlanticus}): A comparison of annulus counts with radiometric ageing.
\emph{Canadian Journal of Fisheries and Aquatic Sciences} \textbf{52},
391--401. doi:\href{https://doi.org/10.1139/f95-041}{10.1139/f95-041}.

\hypertarget{ref-smith_timing_1984}{}
Smith, S.E. (1984) Timing of vertebral-band deposition in
tetracycline-injected leopard sharks. \emph{Transactions of the American
Fisheries Society} \textbf{113}, 308--313.
doi:\href{https://doi.org/10.1577/1548-8659(1984)113\%3C308:TOVDIT\%3E2.0.CO;2}{10.1577/1548-8659(1984)113\textless{}308:TOVDIT\textgreater{}2.0.CO;2}.

\hypertarget{ref-smith_intrinsic_1998}{}
Smith, S.E., Au, D.W. and Show, C. (1998) Intrinsic rebound potentials
of 26 species of Pacific sharks. \emph{Marine and Freshwater Research}
\textbf{49}, 663--678.
doi:\href{https://doi.org/10.1071/MF97135}{10.1071/MF97135}.

\hypertarget{ref-smith_age-validation_2003}{}
Smith, S.E., Mitchell, R.A. and Fuller, D. (2003) Age-validation of a
leopard shark (\emph{Triakis semifasciata}) recaptured after 20 years.
\emph{Fishery Bulletin} \textbf{101}, 194--198.

\hypertarget{ref-taylor_spine-based_2013}{}
Taylor, I.G., Gertseva, V. and Matson, S.E. (2013) Spine-based ageing
methods in the spiny dogfish shark, \emph{Squalus suckleyi}: How they
measure up. \emph{Fisheries Research} \textbf{147}, 83--92.
doi:\href{https://doi.org/10.1016/j.fishres.2013.04.011}{10.1016/j.fishres.2013.04.011}.

\hypertarget{ref-then_evaluating_2014}{}
Then, A.Y., Hoenig, J.M., Hall, N.G. and Hewitt, D.A. (2014) Evaluating
the predictive performance of empirical estimators of natural mortality
rate using information on over 200 fish species. \emph{ICES Journal of
Marine Science: Journal du Conseil}, fsu136.
doi:\href{https://doi.org/10.1093/icesjms/fsu136}{10.1093/icesjms/fsu136}.

\hypertarget{ref-thorson_gear_2009}{}
Thorson, J.T. and Simpfendorfer, C.A. (2009) Gear selectivity and sample
size effects on growth curve selection in shark age and growth studies.
\emph{Fisheries Research} \textbf{98}, 75--84.
doi:\href{https://doi.org/10.1016/j.fishres.2009.03.016}{10.1016/j.fishres.2009.03.016}.

\hypertarget{ref-walker_age_2001}{}
Walker, T.I., Brown, L.P. and Clement, J.G. (2001) Age validation from
tagged school and gummy sharks injected with oxytetracycline, FRDC
Project No. 97/110, Final Report to Fisheries Research and Development
Corporation. 30 pp.

\hypertarget{ref-walker_phenomenon_1998}{}
Walker, T.I., Taylor, B.L., Hudson, R.J. and Cottier, J.P. (1998) The
phenomenon of apparent change of growth rate in gummy shark
(\emph{Mustelus antarcticus}) harvested off southern Australia.
\emph{Fisheries Research} \textbf{39}, 139--163.
doi:\href{https://doi.org/10.1016/S0165-7836(98)00180-5}{10.1016/S0165-7836(98)00180-5}.

\hypertarget{ref-xiao_relationship_2002}{}
Xiao, Y.S. (2002) Relationship among models for yield per recruit
analysis, models for demographic analysis, and age- and time-dependent
stock assessment models. \emph{Ecological Modelling} \textbf{155},
95--125.
doi:\href{https://doi.org/10.1016/S0304-3800(02)00028-5}{10.1016/S0304-3800(02)00028-5}.

\hypertarget{ref-yule_how_2008}{}
Yule, D.L., Stockwell, J.D., Black, J.A., Cullis, K.I., Cholwek, G.A.
and Myers, J.T. (2008) How systematic age underestimation can impede
understanding of fish population dynamics: Lessons learned from a Lake
Superior Cisco stock. \emph{Transactions of the American Fisheries
Society} \textbf{137}, 481--495.
doi:\href{https://doi.org/10.1577/T07-068.1}{10.1577/T07-068.1}.

\hypertarget{ref-zhou_linking_2012}{}
Zhou, S., Yin, S., Thorson, J.T., Smith, A.D. and Fuller, M. (2012)
Linking fishing mortality reference points to life history traits: An
empirical study. \emph{Canadian Journal of Fisheries and Aquatic
Sciences} \textbf{69}, 1292--1301.
doi:\href{https://doi.org/10.1139/f2012-060}{10.1139/f2012-060}.


\end{document}
